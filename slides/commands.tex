\newcommand{\location}{\ell}
\newcommand{\valuation}{\sigma}
\newcommand{\Valuation}{\Sigma}
\newcommand{\update}{\eta}
\newcommand{\atom}{a}
\newcommand{\guard}{\tau}
\newcommand{\cost}{c}
\newcommand{\complexity}{\emph{complexity}}
\newcommand{\landau}{\mathcal{O}}
\newcommand{\constant}{\emph{constant}}

\newcommand{\ValueSet}{\bar{\mathbb{Z}}}

\newcommand{\abs}[1]{\left|#1\right|}

\newcommand{\LSize}{{\mathcal{S}^\sqcup}}
\newcommand{\USize}{{\mathcal{S}^\sqcap}}
\newcommand{\GSize}{{\mathcal{S}^\square}}
\newcommand{\Size}{{(\LSize, \USize)}}
\newcommand{\UTime}{{\mathcal{R}^\sqcap}}
\newcommand{\UCost}{{\mathcal{C}^\sqcap}}
\newcommand{\LLSB}{{\mathcal{S}^\sqcup_l}}
\newcommand{\ULSB}{{\mathcal{S}^\sqcap_l}}
\newcommand{\GLSB}{{\mathcal{S}^\square_l}}
\newcommand{\LSB}{{(\LLSB, \ULSB)}}

\newcommand{\dpre}[1]{{\tilde{d}^{#1}}}
\newcommand{\pret}{r}
\newcommand{\actt}{s}
\newcommand{\prerv}{\gamma}
\newcommand{\actrv}{\alpha}
\newcommand{\outrv}{\beta}
\newcommand{\prestate}{{\tilde{\valuation}}}
\newcommand{\actstate}{\valuation}
\newcommand{\ustate}{{\valuation^\sqcap}}
\newcommand{\lstate}{{\valuation^\sqcup}}
\newcommand{\prel}{{\tilde{\location}}}
\newcommand{\actl}{\location}

\newcommand{\scale}{\emph{scale}}
\newcommand{\incoming}{\emph{in}}
\newcommand{\start}{\emph{start}}
\newcommand{\sign}{\emph{sign}}
\newcommand{\effect}{\emph{effect}}
\newcommand{\SCC}{C}
\newcommand{\timerank}{\emph{timerank}}
\newcommand{\costrank}{\emph{costrank}}
\newcommand{\rv}{\alpha}
\newcommand{\RV}{\text{RV}}
\newcommand{\BoundSet}{\mathcal{B}}
\newcommand{\Program}{\mathcal{P}}
\newcommand{\AtomSet}{\mathcal{A}}
\newcommand{\ConstraintSet}{\mathcal{C}}
\newcommand{\TSet}{\mathcal{T}}
\newcommand{\tvar}{\lambda}
\newcommand{\VSet}{\mathcal{V}}
\newcommand{\TVSet}{\mathcal{TV}}
\newcommand{\PVSet}{\mathcal{PV}}
\newcommand{\AllVarsSet}{(\PVSet \cup \TVSet)}
\newcommand{\LSet}{\mathcal{L}}
\newcommand{\ScaledSum}{\dot{x}}
\newcommand{\pre}{\emph{pre}}
\newcommand{\actV}{\emph{actV}}
\newcommand{\maxO}[1]{\langle#1\rangle}
\newcommand{\maximum}[1]{\max \lbrace #1 \rbrace}
\newcommand{\minimum}[1]{\min \lbrace #1 \rbrace}
\newcommand{\braced}[1]{\lbrace #1 \rbrace}

\newcommand{\usub}{\delta^\sqcap}
\newcommand{\lsub}{\delta^\sqcup}

\newcommand{\usubst}[3]{\lceil #1 \rceil^{#3}_{#2}}
\newcommand{\lsubst}[3]{\lfloor #1 \rfloor^{#3}_{#2}}

\newcommand{\exacteval}[2]{\llbracket #1 \rrbracket_{#2}}
\newcommand{\ueval}[3]{\lceil #1 \rceil^{#3}_{#2}}
\newcommand{\leval}[3]{\lfloor #1 \rfloor^{#3}_{#2}}

\newcommand{\timecomplexityterm}{
  \sup \braced{ k \in \mathbb{N} \mid \exists \valuation_0, (\location, \valuation): \lstate \leq \valuation_0 \leq \ustate \wedge (\location_0, \valuation_0) \rightarrow^k (\location, \valuation) }
}

\newcommand{\timeboundterm}{
  \sup \braced{ k \in \mathbb{N} \mid \exists \valuation_0, (\location, \valuation): \lstate \leq \valuation_0 \leq \ustate \wedge (\location_0, \valuation_0) (\rightarrow^* \circ \rightarrow_t)^k (\location, \valuation) }
}

\newcommand{\sizeboundterm}{
  \braced{ \valuation(v) \mid \exists \valuation_0, (\location, \valuation): \lstate \leq \valuation_0 \leq \ustate \wedge (\location_0, \valuation_0) (\rightarrow^* \circ \rightarrow_\actt) (\location, \valuation)}
}

\newcommand{\monad}[1]{\text{Monad} \langle #1 \rangle}

\newcommand{\usizeboundterm}{\sup \sizeboundterm}
\newcommand{\lsizeboundterm}{\inf \sizeboundterm}

\newcommand{\localsizeboundterm}{
  \braced{\valuation'(v) \mid \exists (\location, \valuation), (\location', \valuation'): \lstate \leq \valuation \leq \ustate \wedge (\location, \valuation) \rightarrow_t (\location', \valuation')}
}

\newcommand{\ulocalsizeboundterm}{\sup \localsizeboundterm}
\newcommand{\llocalsizeboundterm}{\inf \localsizeboundterm}

\newcommand{\costcomplexityterm}{
  \braced{ \sum_{0 \leq i \leq k} \exacteval{\cost(t_i)}{\valuation_i} \mid \exists \valuation_0, k \geq 1: 
    \lstate \leq \valuation_0 \leq \ustate \wedge (\location_0, \valuation_0) \rightarrow_{t_0} (\location_1, \valuation_1) \rightarrow_{t_1} \dots \rightarrow_{t_k} (\location_k, \valuation_k) }
}

\newcommand{\slide}[2]{
  \begin{frame}
    \frametitle{#1}
    \input{frames/#2}
  \end{frame}
}

\makeatletter

\newcount\bt@rangea
\newcount\bt@rangeb

\newcommand\btIfInRange[2]{%
    \global\let\bt@inrange\@secondoftwo%
    \edef\bt@rangelist{#2}%
    \foreach \range in \bt@rangelist {%
        \afterassignment\bt@getrangeb%
        \bt@rangea=0\range\relax%
        \pgfmathtruncatemacro\result{ ( #1 >= \bt@rangea) && (#1 <= \bt@rangeb) }%
        \ifnum\result=1\relax%
            \breakforeach%
            \global\let\bt@inrange\@firstoftwo%
        \fi%
    }%
    \bt@inrange%
}

\newcommand\bt@getrangeb{%
    \@ifnextchar\relax%
        {\bt@rangeb=\bt@rangea}%
        {\@getrangeb}%
}
\def\@getrangeb-#1\relax{%
    \ifx\relax#1\relax%
        \bt@rangeb=100000%   \maxdimen is too large for pgfmath
    \else%
        \bt@rangeb=#1\relax%
    \fi%
}

%%%%%%%%%%%%%%%%%%%%%%%%%%%%%%%%%%%%%%%%%%%%%%%%%%%%%%%%%%%%%%%%%%%%%%%%%%%%%%
%
% \btLstHL<overlay spec>{range list}
%
% TODO BUG: \btLstHL commands can not yet be accumulated if more than one overlay spec match.
% 
\newcommand<>{\btLstHL}[1]{%
  \only#2{\btIfInRange{\value{lstnumber}}{#1}{\color{orange!30}\def\lst@linebgrdcmd{\color@block}}{\def\lst@linebgrdcmd####1####2####3{}}}%
}%
\makeatother

\newcommand{\codeslide}[2]{
  \begin{frame}[t]
    \frametitle{#1}
    \lstinputlisting{../snippets/#2}
  \end{frame}
}

\newcommand{\animatedcodeslide}[3]{
  \begin{frame}[t]
    \frametitle{#1}
    \lstinputlisting[#3]{../snippets/#2}
  \end{frame}
}

\newcommand{\nodeshow}[2]{
  \alt<#1>{#2}{\textcolor{white}{#2}}
}

\newcommand*\circled[1]{
  \FPeval\index{#1-1}
  \FPclip\index\index
  \tikz[baseline=(char.base)]{
    \node[shape=circle,draw,inner sep=1pt] (char) {$\location_\index$};
  }
}

\newcommand{\backupbegin}{
   \newcounter{framenumberappendix}
   \setcounter{framenumberappendix}{\value{framenumber}}
}

\newcommand{\backupend}{
   \addtocounter{framenumberappendix}{-\value{framenumber}}
   \addtocounter{framenumber}{\value{framenumberappendix}} 
}
